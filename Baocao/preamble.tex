\usepackage[utf8]{vietnam}
\usepackage{lmodern}
\usepackage{times}
\usepackage{geometry}
\usepackage{amsmath}
\usepackage{booktabs}
\usepackage{tocloft}
\usepackage{enumitem}
\usepackage{fancyhdr}
\usepackage{indentfirst}
\usepackage{setspace}
\usepackage{array}
\usepackage{hyperref}
\usepackage{float}
\usepackage{multirow}
\usepackage{xspace}
\usepackage{xcolor}
\usepackage{soul}
\usepackage{courier}
\usepackage{titlesec}
\usepackage{etoolbox}
\usepackage{adjustbox}
\usepackage{fancyvrb}
\hypersetup{hidelinks} % Ẩn khung và màu của liên kết
\usepackage{graphicx}
\usepackage{caption}
\usepackage{longtable}
\usepackage{ragged2e} % Cung cấp \justifying
\usepackage{listings}

% =====================
% Thiết lập định dạng CHUNG cho báo cáo 
% Thay đổi ở đây sẽ áp dụng cho toàn bộ tài liệu
% =====================

% 1) Khổ giấy, lề trang
\geometry{top=3.5cm, bottom=3cm, left=3.5cm, right=2cm}

% 2) Cỡ chữ mặc định, giãn dòng, thụt đầu dòng
\renewcommand{\normalsize}{\fontsize{13}{15.6}\selectfont}
\linespread{1.5}
\setlength{\parindent}{1.25cm}
\setlength{\parskip}{0.25cm} 

% 3) Đánh số hình, bảng theo chương/section nếu muốn
\numberwithin{figure}{section}
% \numberwithin{table}{section} % bật nếu cần

% 4) Header/Footer và số trang
\pagestyle{fancy}
\fancyhf{}
\fancyhead[C]{\thepage}
\renewcommand{\headrulewidth}{0pt}

% 5) Mức độ đánh số và mục lục
\setcounter{secnumdepth}{3}
\setcounter{tocdepth}{2}

% 6) Danh sách (list) căn lề, khoảng cách
\setlist[itemize]{align=parleft, leftmargin=, labelsep=0.5em, itemsep=0.5em}
\setlist[enumerate]{align=parleft, leftmargin=, labelsep=0.5em, itemsep=0.5em}

% 7) Tiêu đề các cấp (chapter/section/...) và khoảng cách
\titleformat{\chapter}[display]
{\normalfont\Large\bfseries\filcenter}
{\MakeUppercase{CHƯƠNG \thechapter.}}
{0pt}
{\MakeUppercase}
\titlespacing*{\chapter}{0pt}{-20pt}{20pt}

\titleformat{\section}
{\normalfont\fontsize{14pt}{16.8pt}\selectfont\bfseries}
{\thesection.}
{0.5em}
{}
\titlespacing*{\section}{0pt}{3.5ex plus 1ex minus .2ex}{2.3ex plus .2ex}

\titleformat{\subsection}
{\normalfont\fontsize{14pt}{16.8pt}\selectfont\bfseries\itshape}
{\thesubsection.}
{0.5em}
{}
\titlespacing*{\subsection}{0pt}{3.25ex plus 1ex minus .2ex}{1.5ex plus .2ex}

\titleformat{\subsubsection}
{\normalfont\fontsize{14pt}{16.8pt}\selectfont}
{\thesubsubsection.}
{0.5em}
{}
\titlespacing*{\subsubsection}{0pt}{3ex plus 1ex minus .2ex}{1ex plus .2ex}

% 8) Mục lục, danh mục hình/bảng
% --- Định dạng cho TẤT CẢ các cấp ---
\renewcommand{\cftchapleader}{\cftdotfill{\cftdotsep}} 
\renewcommand{\cftsecleader}{\cftdotfill{\cftdotsep}}
\renewcommand{\cftsubsecleader}{\cftdotfill{\cftdotsep}}

% --- Định dạng cho cấp CHƯƠNG (Chapter) và các mục tương đương ---
\renewcommand{\cftchapfont}{\bfseries}      
\renewcommand{\cftchappagefont}{\bfseries} 
\renewcommand{\cftchappresnum}{CHƯƠNG } 
\renewcommand{\cftchapaftersnum}{.}
\setlength{\cftchapnumwidth}{2.75cm}

% --- Định dạng cho cấp MỤC (Section) và MỤC CON (Subsection) ---
\setlength{\cftsecindent}{0pt}
\setlength{\cftsubsecindent}{0pt}
\setlength{\cftsecnumwidth}{1cm}
\setlength{\cftsubsecnumwidth}{1cm}

% --- Tên và tiêu đề các danh mục ---
\renewcommand{\listtablename}{DANH MỤC BẢNG BIỂU}
\renewcommand{\listfigurename}{DANH MỤC HÌNH ẢNH}
\newcommand{\ReportLargeCentered}[1]{\centerline{\normalfont\fontsize{16}{19.2}\selectfont\bfseries #1}}
\renewcommand{\contentsname}{\ReportLargeCentered{MỤC LỤC}}
\newcommand{\ReportListOfFiguresTitle}{\ReportLargeCentered{DANH MỤC HÌNH ẢNH}}
\newcommand{\ReportListOfTablesTitle}{\ReportLargeCentered{DANH MỤC BẢNG BIỂU}}

% --- Khoảng cách trước/sau tiêu đề các danh mục ---
\setlength{\cftbeforetoctitleskip}{-20pt}
\setlength{\cftaftertoctitleskip}{20pt}
\setlength{\cftbeforeloftitleskip}{-20pt}
\setlength{\cftafterloftitleskip}{20pt}
\setlength{\cftbeforelottitleskip}{-20pt}
\setlength{\cftafterlottitleskip}{20pt}
% 9) Chú thích hình/bảng
\captionsetup[table]{position=top, skip=0pt, font=normalsize, labelfont=bf}
\captionsetup[figure]{position=bottom, skip=10pt, font=normalsize, labelfont=bf}

% 10) Tiện ích: lệnh in tiêu đề danh mục theo chuẩn đã định nghĩa
\newcommand{\PrintListOfFigures}{%
\renewcommand{\listfigurename}{\ReportListOfFiguresTitle}%
\listoffigures
}

% =====================
% Định nghĩa lệnh tùy chỉnh cho tiêu đề
\newcommand{\headingone}[1]{
  {\noindent\fontsize{16pt}{16.5pt}\selectfont \bfseries #1}
}

% =====================
% HẾT cấu hình chung và bổ sung
% =====================